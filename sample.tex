\documentclass[autodetect-engine,
dvi=dvipdfmx,ja=standard,
               a4j,11pt]{bxjsarticle}

%% ======== パッケージ類 ========================================== %%
\usepackage{multirow}
\usepackage{array}
\usepackage{geometry}
\usepackage{bytefield}

\geometry{reset,paperwidth=210truemm,paperheight=297truemm,
          hmargin=25truemm,top=20truemm,bottom=25truemm,footskip=10truemm}

\usepackage{graphicx}
\usepackage{fancyvrb}
\fvset{numbers=left, numbersep=6pt, fontsize=\small, xleftmargin=10mm}
\usepackage{booktabs}
\usepackage{caption}
\usepackage{siunitx}
\sisetup{detect-all, table-format=3.0}

%% ======== タイトルなど ========================================== %%
\title{SMLを使用したTodoリストアプリケーションの実装}

\author{学生番号: 09B23549\\
        提出者:  中嶋 空偉\\
        E-mail: pj9y6y4w@s.okayama-u.ac.jp}
\date{
      締切日: 2025年5月26日 (月)}

\begin{document}
\maketitle

\section{プログラムの概要}
本レポートでは,Standard ML(SML)を使用して実装したTodoリストアプリケーションについて述べる.
このアプリケーションは,タスクの管理,追加,更新,削除などの基本的な機能を提供し,
授業で学んだSMLの機能を活用して実装されている.

\section{作ろうと思った背景}
新しいプログラミング言語を学ぶ際,私は常にHelloWorldプログラムを作成し,
次にTodoリストプログラムを実装するという習慣がある.
HelloWorldは基本的すぎるため,今回はTodoリストの実装に焦点を当てた.

SMLは関数型プログラミング言語であり,パターンマッチングや再帰関数,
高階関数などの特徴的な機能を持つ.これらの機能を活用して,
実用的なアプリケーションを実装できるかという挑戦的な目標を設定した.

\section{実装したい機能}
本アプリケーションでは,以下のCRUD(Create, Read, Update, Delete)機能を実装する:

\begin{itemize}
    \item Create: 新しいTodoアイテムの作成
    \item Read: Todoリストの表示,完了/未完了のフィルタリング
    \item Update: Todoアイテムの更新,状態の変更
    \item Delete: Todoアイテムの削除
\end{itemize}

\section{SMLプログラムへの実装方法}
\subsection{データ構造の設計}
まず,Todoアイテムを表現するためのデータ型を定義した:

\begin{verbatim}
datatype status = Pending | Completed;

datatype todo = Todo of {
    title: string,
    description: string,
    dueDate: string,
    status: status
};

type todoList = todo list;
\end{verbatim}

\subsection{主要な関数の実装}
\subsubsection{作成機能}
新しいTodoアイテムを作成する関数を実装した:

\begin{verbatim}
fun createTodo (title: string, description: string, 
               dueDate: string) : todo =
    Todo {
        title = title,
        description = description,
        dueDate = dueDate,
        status = Pending
    };
\end{verbatim}

\subsubsection{表示機能}
Todoリストを表示する関数を実装した:

\begin{verbatim}
fun displayList (list: todoList) : unit =
    let
        fun display (l: todoList, index: int) : unit =
            case l of
                [] => ()
              | x::xs => (
                    print ("[" ^ Int.toString index ^ "] ");
                    displayTodo x;
                    display (xs, index + 1)
                )
    in
        display (list, 0)
    end;
\end{verbatim}

\subsubsection{更新機能}
Todoアイテムの状態を更新する関数を実装した:

\begin{verbatim}
fun updateStatus (list: todoList, index: int, 
                 newStatus: status) : todoList =
    let
        fun update (l: todoList, i: int, 
                   acc: todoList) : todoList =
            case (l, i) of
                ([], _) => rev acc
              | (Todo {title, description, dueDate, status}::xs, 0) =>
                    rev acc @ (Todo {
                        title=title, 
                        description=description, 
                        dueDate=dueDate, 
                        status=newStatus
                    }::xs)
              | (x::xs, n) => update (xs, n-1, x::acc)
    in
        update (list, index, [])
    end;
\end{verbatim}

\subsubsection{削除機能}
Todoアイテムを削除する関数を実装した:

\begin{verbatim}
fun removeTodo (list: todoList, index: int) : todoList =
    let
        fun remove (l: todoList, i: int, 
                   acc: todoList) : todoList =
            case (l, i) of
                ([], _) => rev acc
              | (x::xs, 0) => rev acc @ xs
              | (x::xs, n) => remove (xs, n-1, x::acc)
    in
        remove (list, index, [])
    end;
\end{verbatim}

\section{テスト結果}
プログラムの動作確認を行った結果,以下の機能が正常に動作することを確認した:

\begin{itemize}
    \item Todoアイテムの作成と表示
    \item 状態の更新(完了/未完了の切り替え)
    \item アイテムの削除
    \item 完了済み/未完了のフィルタリング
\end{itemize}

テストの実行結果は以下の通りである:

\begin{verbatim}
=== すべてのTodo ===
[0] タイトル: テスト
説明: テスト用のTodo
期限: 2024-03-21
状態: 完了
-------------------
[1] タイトル: テスト2
説明: 2つ目のテスト
期限: 2024-03-22
状態: 未完了
-------------------

=== 完了済みのTodo ===
[0] タイトル: テスト
説明: テスト用のTodo
期限: 2024-03-21
状態: 完了
-------------------

=== 未完了のTodo ===
[1] タイトル: テスト2
説明: 2つ目のテスト
期限: 2024-03-22
状態: 未完了
-------------------
\end{verbatim}

\section{まとめ}
本レポートでは,SMLを使用したTodoリストアプリケーションの実装について述べた.
授業で学んだSMLの機能を活用し,実用的なアプリケーションを実装することができた.
特に,パターンマッチングや再帰関数,高階関数などの機能を効果的に使用することで,
簡潔で保守性の高いコードを実現することができた.
授業で習った時は,シンプルなものしか作れないと感じていたが,実用的なアプリケーションを実装することができたことに満足している.

今後の課題として,以下の機能の追加が考えられる:
\begin{itemize}
    \item 期限によるソート機能
    \item ファイルへの保存/読み込み機能
    \item より詳細な検索機能
    \item エラー処理の強化
\end{itemize}

\end{document}
